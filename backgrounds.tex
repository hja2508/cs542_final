
\section{Background}

\subsection{Belief Propagation}
Belief Propagation, also known as \textbf{sum-product message passing} is a
message passing algorithm for performing inference on graphical models. 
Each edge associates a message with each direction, which is updated using incoming messages. 
Messages are updated until convergence, and then beliefs are calculated from each node. 
%\subsection{Weighted matching and its LP relaxation}
%\subsection{Max-product for weighted matching}
\subsection{Simplified Max-product for weighted matching}
\begin{itemize}
\item

%\begin{displaymath}\int^b_af(t)dt = G(b) - G(a).\end{displaymath}
%\begin{equation}\sum_{i=0}^{\infty}x_i=\int_{0}^{\pi+2} f\end{equation}

\textbf{(INIT)} Set $t$=0 and initialize each $a_{i \rightarrow j}^0$ = 0
\item
\textbf{(ITER)} Iteratively compute new messages until convergence as follows: ($y_+$ = max(0, $y$))
\begin{displaymath}
a_{i \rightarrow j}^{t+1} = max_{k \in N(i)-j} (w_{ik} - a_{k \rightarrow i}^{t})_+
\end{displaymath}
\item
\textbf{(ESTIM)} Upon convergence, output estimate $\hat{x}_{(i,j)}$ is, respectively, >, <, or = $w_{ij}$.
\end{itemize}

\subsection{ZooKeeper}
ZooKeeper is a software project of the Apache Software Foundation, providing an open source distributed configuration service, synchronization service, and naming registry for large distributed systems. 

ZooKeeper nodes store their data in a hierarchical name space, much like a file system or a tree data structure. Clients can read and write from/to the nodes and in this way have a shared configuration service. 

ZooKeeper is used in many applications such as naming service, configuration management, leader election, etc.


