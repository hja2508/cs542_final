
\section{Background}

\subsection{Belief Propagation}
Belief Propagation is a message passing algorithm
for performing inference on graphical models. It is common used in artificial intelligence and information theory and has demonstrated empirical success in numerous applications.

On the graph, messages are associated on each edge with each direction and updated until convergence.
Each node has a \textbf{belief}, which is calculated from incoming messages. 
Those beliefs are used as decision variable of nodes and edges. 

%\subsection{Weighted matching and its LP relaxation}
%\subsection{Max-product for weighted matching}
\subsection{Simplified Max-product for weighted matching~\cite{BPLPmatching}}
Let $m_{i \rightarrow j}$ denote the messsage at time $t$, node $i$ and edge $e in E_i$. 
For every pair of neighbors $i$ and $j$, let $e = (i, j)$ be the edge connecting the two,
and define 
\begin{displaymath}
a_{i \rightarrow j}^t = log(\frac{m_{i \rightarrow e}^t[0]}{m_{i \rightarrow e}^t[1]})
\end{displaymath}

\begin{itemize}
\item

%\begin{displaymath}\int^b_af(t)dt = G(b) - G(a).\end{displaymath}
%\begin{equation}\sum_{i=0}^{\infty}x_i=\int_{0}^{\pi+2} f\end{equation}

\textbf{(INIT)} Set $t$=0 and initialize each $a_{i \rightarrow j}^0$ = 0
\item
\textbf{(ITER)} Iteratively compute new messages until convergence as follows: ($y_+$ = max(0, $y$))
\begin{displaymath}
a_{i \rightarrow j}^{t+1} = max_{k \in N(i)-j} (w_{ik} - a_{k \rightarrow i}^{t})_+
\end{displaymath}
\item
\textbf{(ESTIM)} Upon convergence, output estimate $\hat{x}_{(i,j)}$ is, respectively, >, <, or = $w_{ij}$.
\end{itemize}

It guarantees convergence when the extreme points of the matching LP polytope are integral. In other words, if LP doesn't have any fractional optima. 


\subsection{Hungarian algorithm~\cite{wiki_hungarian}}
The Hungarian algorithm is a combinatorial optimization algorithm that solves the assignment problem in polynomial time. 
The time complexity of the original algorithm was $O(n^4)$, modified to achieve $O(n^3)$ running time. 

If we formulate the problem using a bipartite graph, $G=(S, T;E)$ with $n$ vertices on one side ($S$) and same number of vertices on the other side ($T$), and each edge has a nonnegative weight $w(i, j)$, the way to find a perfect matching with minimum cost is as below. (Edges with 0 weight are added to make the graph complete in advance)

Let $y:(S \cup T) \mapsto \Re$ a \textbf{potential} if $y(i) + y(j) \leq c(i,j)$ for each $i \in S, j \in T$.
The value of potential $y$ is $\sum_{v \in S \cup T}^{y(v)}$. The Hungarian method finds a perfect matching of tight edges: an edge $ij$ is called tight for a potential $y$ if $y(i) + y(j) = c(i,j)$. Let $G_y$ denote the subgraph of tight edges. The cost of a perfect matching in $G_y$ equals the value of $y$.

\subsection{Blossom algorithm~\cite{wiki_blossom}}
The blossom algorithm is an algorithm for constructing maximum matchings on graphs.
The matching is constructed by iteratively improving an initial empty matching along augmenting paths in the graph. 
Micali and Vazirani~\cite{blossom} showed an algorithm that constructs maximum matching in $O(|E||V|^{1/2})$ time.

